%
%  This is ``front matter'' for the thesis.
%
%  Regarding ``References'' below:
%      KEY    MEANING
%      PU     ``A Manual for the Preparation of Graduate Theses'',
%             The Graduate School, Purdue University, 1996.
%      PU8    ``A Manual for the Preparation of Graduate Theses'',
%             Eighth Revise Edition, Purdue University.
%      TCMOS  The Chicago Manual of Style, Edition 14.
%      WNNCD  Webster's Ninth New Collegiate Dictionary.
%
%  Lines marked with "%%" may need to be changed.
%

  % Statement of Thesis/Dissertation Approval Page
  % This page is REQUIRED.  The page should be numbered page ``ii''
  % and should NOT be listed in your TABLE OF CONTENTS.
  % References: PU8 ordinal pages 5 and 29.
  % The web page https://engineering.purdue.edu/AAE retrieved on
  % January 8, 2017 had "School of Aeronautics and Astronautics"---that
  % is used instead of "Department af Aeronautics and Astronautics"
  % below.
\begin{statement}
  \entry{Dr.~Christopher S. Goldenstein, Chair}{School of Mechanical Engineering}
  \entry{Dr.~Gregory M. Shaver}{School of Mechanical Engineering}
  \entry{Dr.~Terrence R. Meyer}{School of Mechanical Engineering}
  \approvedby{Dr.~Jay P. Gore}{Director of Graduate Programs}
\end{statement}

  % Dedication page is optional.
  % A name and often a message in tribute to a person or cause.
  % References: PU 15, WNNCD 332.
\begin{dedication}
For my parents
\end{dedication}

\begin{acknowledgments}
Throughout my academic journey, I have been so lucky to receive support from my family, friends, colleagues and mentors. Without their help I couldn't have achieved all that I have.

I would like to first give my most sincere gratitude to my advisor, Prof. Chris Goldenstein, for bringing me to Purdue and providing me with great research opportunities throughout the time. Not only has he offered patient guidance with incredible inspiration and insight, but more importantly, he encouraged and taught me to keep being diligent and pursuing great excellence in research and life. Although I am still far away from that, I will keep it in my mind for the rest of my journey of life. I appreciate the opportunity Prof. Gregory Shaver has given me to take measurements in his engine in the Herrick Laboratories and for serving on my committee. I also thank Prof. Terrence Meyer for serving on my committee. 

I have also spent a great time exploring new things with my dear friends and working with fantastic fellow engineers. The students of the Goldenstein Group (Ryan Tancin, Garrett Mathews, Morgan Ruesch and Austin McDonald) share a great deal of incredible ideas and knowledge. I would like to thank Garrett Mathews for his help and support in my experiment. I also appreciate Dheeraj Gosala for helping me with all kinds of work in Herrick.

Finally but most importantly, I would like to thank my dear parents for their endless love and encouragement. They are always ready for supporting me financially and mentally. My mother created a home of love, and kept imparting her wisdom and a great number of life skills and interpersonal skills to me. My father, also majoring in mechanical engineering, always reminded me of working hard and being aware of what matters most at the present stage. He provided me a lot of visions in the industries, and always encouraged me to go beyond my comfort zone and take up new challenges.

\end{acknowledgments}

  % The preface is optional.
  % References: PU 16, TCMOS 1.49, WNNCD 927.
%\begin{preface}
%  This is the preface.
%\end{preface}

  % The Table of Contents is required.
  % The Table of Contents will be automatically created for you
  % using information you supply in
  %     \chapter
  %     \section
  %     \subsection
  %     \subsubsection
  % commands.
  % Reference: PU 16.
\tableofcontents

  % If your thesis has tables, a list of tables is required.
  % The List of Tables will be automatically created for you using
  % information you supply in
  %     \begin{table} ... \end{table}
  % environments.
  % Reference: PU 16.
\listoftables

  % If your thesis has figures, a list of figures is required.
  % The List of Figures will be automatically created for you using
  % information you supply in
  %     \begin{figure} ... \end{figure}
  % environments.
  % Reference: PU 16.
\listoffigures

  % List of Symbols is optional.
  % Reference: PU 17.
%\begin{symbols}
%  $m$& mass\cr
%  $v$& velocity\cr
%\end{symbols}

  % List of Abbreviations is optional.
  % Reference: PU 17.
%\begin{abbreviations}
%  abbr& abbreviation\cr
%  bcf& billion cubic feet\cr
%  BMOC& big man on campus\cr
%\end{abbreviations}

  % Nomenclature is optional.
  % Reference: PU 17.
\begin{nomenclature}
  $A$& Integrated absorbance\cr
  CAD& Computer aided design\cr
  DA& Direct absorption\cr
  DFB& Distributed-feedback\cr
  DPF& Diesel particulate filter\cr
  $E^"$& Lower-state energy\cr
  $f_m$& Modulation frequency\cr
  $f_s$& Scan frequency\cr
  FTS& Fourier Transform Spectroscopy\cr
  FWHM& Full-width at half-maximum\cr
  $G$& Detector gain\cr
  $h$& Planck's constant\cr
  HCCI& Homogeneous-charge compression ignition\cr
  HWHM& Half-width at half-maximum\cr
  IC& Internal combustion\cr
  $I_0$& Incident laser light intensity\cr
  $I_t$& Transmitted laser light intensity\cr
  IR-LAS& Infrared laser-absorption spectroscopy\cr
  $k$& Boltzmann constant\cr
  $k_\nu$ & Absorption coefficient\cr
  $L$& Path length through absorbing gas\cr
  LOS& Line-of-sight\cr
  $M$& Molecular weight\cr
  $n$& Number density\cr
  OSSP& optical spark plug probe\cr
  $P$& Pressure\cr
  $Q$& Partition function\cr
  QCL& Quantum-cascade laser\cr
  $R$& Two-color ratio of integrated areas\cr
  $S$& Transition linestrength\cr
  SCR& Selective catalytic reduction\cr
  SMF& Single-mode fiber\cr
  SNR& Signal-to-noise ratio\cr
  $T$& Temperature\cr
  $T_0$& Reference temperature\cr
  TDLAS& Tunable diode laser absorption spectroscopy\cr
  WMS& Wavelength modulation spectroscopy\cr
  $\alpha$& Absorbance\cr
  $\gamma_k$& Collisonal-broadening coefficient\cr
  $\nu$& Optical frequency in $cm^{-1}$\cr
  $\nu_0$& Line center frequency\cr
  $\phi$& Line shape function\cr
  $\chi$& Mole fraction of absorbing species\cr
  $\psi$& Phase shift between laser intensity and frequency modulation\cr
  $\delta$& Pressure-shift coefficient\cr
  $\Delta\nu_C$& Collisional-broadening FWHM\cr
  $\Delta\nu_D$& Doppler-broadening FWHM\cr
  
\end{nomenclature}

  % Glossary is optional
  % Reference: PU 17.
%\begin{glossary}
%  chick& female, usually young\cr
%  dude& male, usually young\cr
%\end{glossary}

  % Abstract is required.
  % Note that the information for the first paragraph of the output
  % doesn't need to be input here...it is put in automatically from
  % information you supplied earlier using \title, \author, \degree,
  % and \majorprof.
  % Reference: PU 17.
\begin{abstract}
\noindent \qquad Tunable diode-laser-absorption spectroscopy (TDLAS) sensors have been one widely-used laser-diagnostic technique that offers great potential for non-intrusive, time-resolved and multi-parameter sensing in combustion systems. These sensors have been used for performance testing, model validation and feedback
control of combustors \cite{Goldenstein2017,Ma2013,Caswell2013,Stritzke2015,Witzel2013,Whitney2011,Makowiecki2017,Rieker2009b,Li2011}. During operation, monochromatic laser light with a specific wavelength is transmitted through the test gas and collected on a photodetector. Gas conditions such as temperature and composition are then inferred by comparing the measured amount of light that is absorbed with that predicted by spectroscopic
models.

In this thesis, the design and demonstration of a compact single-ended laser-absorption spectroscopy (SE-LAS) sensor for measuring temperature and $H_2O$ in high-temperature combustion gases is presented. The primary novelty of this work lies in the design, demonstration, and evaluation of a sensor architecture which uses a single lens to provide single-ended, alignment-free measurements of gas properties in a combustor without windows. The sensor is demonstrated to be capable of sustained operation at temperatures up to at least 625 $K$ and is capable of withstanding direct exposure to high-temperature ($\approx$1000 $K$) flame gases for long durations (at least 30 min) without compromising measurement quality. The sensor employs a fiber bundle and a 6-mm diameter AR-coated lens mounted in a 1/8" NPT-threaded stainless-steel body to collect laser light that is backscattered off native surfaces (e.g., a combustor wall). Distributed-feedback (DFB) tunable diode lasers (TDLs) with a wavelength near 1392 nm and 1343 nm were used to interrogate well characterized $H_2O$ absorption transitions using wavelength-modulation spectroscopy (WMS) techniques. The sensor is demonstrated with measurements of gas temperature and $H_2O$ mole fraction in a propane-air burner with a measurement bandwidth up to 25 $kHz$. In addition, this work presents an improved wavelength-modulation-spectroscopy spectral-fitting technique which reduces computational time by a factor of 100 compared to previously developed techniques.
\end{abstract}